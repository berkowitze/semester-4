\documentclass[main.tex]{subfiles}

\begin{document}

\section{Chapter 1: Introduction to Probability}
(Chapter 2 Wackerly)

Jan 27, 2017

\vspace{3mm}

\textbf{Probability}: Try to predict the outcome of a process involving randomness
\textbf{Statistics}: Try to understand the randomness causing the outcome of a process

Aim: Introduce basic concepts of probability theory.

\subsection{Probability Space}

\textbf{Definitions}

\begin{itemize}

\item \textbf{Starting point}: random experiment (more than one possible outcome).

\item \textbf{Sample space}: a set containing all possible outcomes. We denote this set with $S$ (sometimes $\Omega$).

\item \textbf{Sample Point}: each possible outcome $x\in S$.

\item \textbf{Event}: Any subset of $S$, $E\subset S$
\end{itemize}

Example: Flip two coins.
\begin{itemize}
    \item Outcomes: HH, HT, TH, TT
    \item Sample Space $S$ = $\left\{HH, HT, TH, TT\right\}$
    \item Sample Points: $x_1 = $HT is a sample point
    \item Event Examples:
    \begin{itemize}
        \item $E_1 = \left\{\textrm{HH, HT, TH}\right\}$ (don't flip tails twice)
        \item $E_2 = \left\{\textrm{TH, HT}\right\}$
        \item $E_3 = \left\{\right\} = \varnothing$
        \item $E_4 = \textrm{ all outcomes}$
    \end{itemize}
\end{itemize}

A \textbf{probability} tells you how likely an \textit{event} is.

Something that always happens $\implies P = 1$
Something that never happens $\implies P = 0$

Definition: A probability $\mathbb{P}$ in a sample space $S$ is a function that assigns to each event a non-negative number.
We require that:
\begin{enumerate}
    \item $0 \le \mathbb{P}(E) \le 1$ for every event
    \item $\mathbb{P}(S) = 1$
    \item With $E_i$, $E_j$ ($i \neq j$), if $E_i \cap E_j = \varnothing$ then $\mathbb{P}(E_1\cup E_2\cup...) = \mathbb{P}(E_1) + \mathbb{P}(E_2) + ...$
\end{enumerate}

Remarks:
\begin{enumerate}
    \item $\mathbb{P}(E_1\cup E_2) = \mathbb{P}(E_1) + \mathbb{P}(E_2)$ if $E_1 \cup E_2 = \varnothing$
    \item If $\bar{E} = S \setminus E$ (Complement of event) then $\mathbb{P}(\bar{E}) = 1 - \mathbb{P}(E)$
    \item $\mathbb{P}(\varnothing) = 0$
\end{enumerate}

A probability space is a sample space $S$, together with a probability on $S$.

Back to the example:

Suppose the coin is fair (each outcome is equally likely).

Then $\mathbb{P}(HH) = \mathbb{P}(HT) = \mathbb{P}(TH) = \mathbb{P}(TT) = 1/4$
\begin{itemize}
    \item $\mathbb{P}(E_1) = \mathbb{P}(\left\{\textrm{HH, HT, TH}\right\}) = 3/4$
    \item $\mathbb{P}(E_2) = \mathbb{P}(\left\{\textrm{HT, TH}\right\}) = 1/2$
    \item $\mathbb{P}(E_3) = \mathbb{P}(\left\{\right\}) = 0$
    \item $\mathbb{P}(E_4) = \mathbb{P}(\left\{\textrm{All events}\right\}) = 1$
\end{itemize}

\end{document}
