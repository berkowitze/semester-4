\documentclass[main.tex]{subfiles}

\begin{document}

\section{Bifurcation Theory}

Jan 27, 2017

\vspace{3mm}
\subsection{Damped bead on a rotating hoop}
Overdamped bead on a rotating hoop (\ref{fig:hoop-and-bead}). Hoop immersed in viscous fluid.

Angular velocity $\omega$.

Use Newton's Law (sum of forces = mass $\times$ acceleration)

Forces:
\begin{itemize}
    \item Gravity ($mg$)
    \item Centrifugal ($m\omega^2 r\sin{\phi}$)
\end{itemize}

Decompose forces into tangential components (other components won't move bead).

\begin{itemize}
    \item Gravity Component: $mg\sin{\phi}$
    \item Centrifugal Component: $m\omega^2r\sin{\phi}\cos{\phi}$
    %\item Frictional Component:
\end{itemize}

Therefore:
\begin{align}
ma &= m\omega^2r\sin{\phi(t)}\cos{\phi(t)} - mg\sin{\phi(t)} \\
&= mr\frac{d^2\phi}{dt^2}
\end{align}

Including damping:

\begin{align}
mr\frac{d^2\phi}{dt^2} &= m\omega^2 r\sin{\phi(t)}\cos{\phi(t)} \\
                       &- mg\sin{\phi(t)} - b\frac{d\phi}{dt}
\end{align}

Assuming $b >> 1$ compared to other parameters (damping dominates acceleration forces) $\implies mr\frac{d^2\phi}{dt^2} \rightarrow 0$

\begin{align}
\dot{\phi} &= \frac{mg}{b}\sin{\phi}\left(\frac{r\omega^2}{g}\cos{\phi} - 1\right) \\
           &= a\sin{\phi}\left(\mu\cos{\phi} - 1\right)
\end{align}
with $a = mg/b$ and $\mu=\frac{r\omega^2}{g}$

Goal is to find how $\mu$ effects the dynamics.

\subsection{Review of ODE stuff}

\begin{enumerate}
    \item $\frac{du}{dt} = f(v), v\in\mathbb{R}, f: \mathbb{R} \mapsto \mathbb{R}$

    \item $v(t)$ is position of a particle at time $t$

    \item $f(v)$ is the velocity of the particle at position v
\end{enumerate}

\textbf{Existence and Uniqueness Theorem:}

If $f$ is twice differentiable, and fix $v_0\in\mathbb{R}$. Then the equation $\frac{du}{dt}=f(u)$ with $v(0) = v_0$ has a unique solution on an interval containing $t=0$.

Geometric Viewpoint:

fig
\begin{itemize}

\item If $u_\star$ is such that $f(u_\star)=0$, then $u(t)=u_\star\forall t$ is a (and therefore the) solution.

Proof: $\frac{du}{dt} == f(u(t))$

$u(t)$ is a constant function, therefore it's derivative is 0. $f(u(t)) = 0$ by assumption.

The points $u_\star$ are called equilibra points.

\item If $f(u) > 0 \implies \frac{du}{dt} = f(u) > 0 \implies u(t)$ increases as t increases $\implies u(t)$ moves to the right.

\item If $f(u) < 0 \implies \frac{du}{dt} = f(u) < 0 \implies u(t)$ decreases as t increases $\implies u(t)$ moves to the left.

\end{itemize}

From figure, $v_1, v_3$ are \textbf{stable equilibra}: Solutions with initial conditions near $v_1$ (or $v_3$) will converge to that $v_1$ (or $v_3$)

$v_2$ is an \textbf{unstable equilibra}: solutions with initial conditions near $u_2$ but not exactly on $v_2$ move away as $t$ increases.

We find these equilibra by solving $f(u) = 0$
\end{document}
