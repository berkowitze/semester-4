\documentclass[main.tex]{subfiles}

\begin{document}

\hrulefill{}

Feb 3, 2017

\vspace{3mm}

Solving explicitly is possible with simple domains like rectangles or disks, but with more complex domains it is often impossible or very difficult. We can still determine properties though.

Suppose $\Delta u = 0$. Then $u$ has the \textbf{mean value property}.

\textbf{Mean value property} where $B$ is a ball.
\begin{equation}
\label{eq:mean-value}
u(\vec{x_0}) = \frac{1}{\textrm{Vol }B(x_0, r)}\int_{B(\vec{x_0}, r)} u\,dV = \frac{1}{\textrm{Area }\partial B(\vec{x}_0, r)} \int_{\partial B(\vec{x_0}, r)} u(x)\,dA
\end{equation}

Which is true for \textbf{all} $\vec{x_0}$ and all $r > 0$.

$$\frac{1}{\textrm{Area }\partial B(\vec{x}_0, r)} \int_{\partial B(0, 1)} u(\vec{x_0} + r\vec{n})\,dA(\vec{n})\cdot \frac{\textrm{area}(\partial B(x_0, r))}{\textrm{area}(\partial B(0, 1))}$$

Now take $\frac{d}{dr}$ which will give $0$:

$$\int_{\partial B(\vec{0}, 1)} \Del \vec{u}(x_0 + r\vec{n}) \cdot \,dA\,dV = \int_{B(\vec{0}, 1)} \divg{\Del u(x_0 + rn)} \, dV = 0$$

This follows from the fact that $\divg{\Del u} = \Delta u = 0$

Implications:

\subsection{Boundary minima and maxima Implication}
Unless $u$ is a constant, then the max and min of $u$ are achieved only on the boundary $\partial \Omega$.

This makes sense with steady state heat equation.

\subsubsection{Continuity Implication}

Suppose $u$ is a $C^2$ solution of $\Delta u = 0$. Then $u$ is automatically $C^\infty$.

\subsubsection{Uniqueness implication}

Suppose $u, v$ both solve:
$$
\begin{cases}
    \Delta u = f & \textrm{in } \Omega \\
    u = g & \textrm{on } \partial \Omega
\end{cases}
$$
Then $u = v$ (uniqueness of solutions).

Proof:

Let $w = u - v$

Then $$
\begin{cases}
\Delta w = 0 & \textrm{in } \Omega \\
w = 0 & \textrm{on } \partial \Omega
\end{cases}$$

This means max and min are zero, and max min are on boundary, so $w = 0$ in $\Omega$. Then $u = v$.

\end{document}
