\documentclass[main.tex]{subfiles}

\begin{document}

\hrulefill{}

Feb 1, 2017

\vspace{3mm}

\subsection{Heat equation in higher dimensions}

Space variables in $\mathbb{R}^d$ such as $(x, y)$ or $(x, y, z)$

Denote as $\vec{x} = (x_1, \dotsb, x_d)$

Working in domain $\Omega \subset \mathbb{R}^d$

In one dimension (eq \#\ref{eq:heat}), $u_t + \phi_x = f$ with $\phi = -ku_x$ giving $u_t - ku_{xx} = f$.

Found this by taking the derivative of the total amount: $\frac{d}{dt}\int_a^b u\,dx = \phi(a) - \phi(b) + \int_a^b f\,dx$

Let $B$ be a solid ball inside of $\Omega$.

$$B = \vec{B}(\vec{x}, r) = \{\vec{y} \in \mathbb{R}^d \textrm{ : } \left|\vec{x} - \vec{y}\right| < r\}$$

$$\int_B u\,dV \textrm{ proportional to heat energy.}$$

Need to know flux in or out of the ball.
$$\frac{d}{dt}\int_B u\,dV = \int_B f\,dV - \int_{\partial B} \vec{\phi}\cdot \vec{n}\,dA$$

Here,
\begin{itemize}
    \item $\partial B$ = boundary of $B = \{\vec{y} \in \mathbb{R}^d : \left|\vec{x} - \vec{y}\right| = r\}$
    \item $\vec{\phi} = \left(\textrm{ flux in direction parallel to } x_1 \textrm{, \ldots, flux in direction parallel to } x_d\right)$
    \item $\vec{n} = \textrm{ outward unit normal}$
\end{itemize}

Therefore, $\vec{\phi}\cdot \vec{n}$ = rate of flux per unit area across surface perpendicular to $\vec{n}$.

Use divergence theorem:
$$\int_{\partial B} \vec{\phi} \vec{n}\,dA = \int_B \divg{\phi}\, dV$$

What is the divergence here?

$$\vec{\phi} = \left(\phi_1, \phi_2, \ldots, \phi_d\right) \implies \divg{\vec{\phi}} = \left(\partial_{x_1}\phi_1 + \partial_{x_2} + \cdots + \partial_{x_d}\phi_d\right)$$

Using this:

$$\frac{d}{dt} \int_B u\,dV = \int_B f - \divg{\vec{phi}}\,dV = \int_B u_t\, dV$$

Equating integrands (assuming continuity):

$$u_t = f - \divg{\phi}$$

Using Fourier's law of conduction, $\vec{\phi} = -k\nabla u$.

Recall that $\divg{\nabla u} = \Delta u$:
\begin{equation}
    \label{eq:heat-higher-dimensions}
    u_t - k\Delta u = f
\end{equation}

\subsubsection{Example}
Assume $\Omega$ does not include boundary.
$$
\begin{cases}
    u_t - \Delta u = f & (x, t) \in \Omega\times \interval{0}{\infty} \\
    u = g & (x, t) \in \partial \Omega \times \interval{0}{\infty} \\
    u = u_0 & (x, t) \in \Omega \times \{0\}
\end{cases}
$$


\subsection{Laplace Equation}
Suppose $g, f$ are independent of time. After long time, expect a steady state ($u(\vec{x}, t) = w(\vec{x})$) where $w(x)$ solves the PDE and the boundary condition, but not the initial condition. Because of this $u_t = 0$.

Therefore, $\vec{w}(x)$ solves
$$
\begin{cases}
    -k\Delta w = f & \vec{x} \in \Omega \\
    w = g & \vec{x} \in \partial\Omega \\
\end{cases}
$$
This yields two functions:

Poisson equation:
$$\Delta w = -\frac{1}{k} f$$
Laplace equation:
$$\Delta w = 0$$
\end{document}
